% Submissions may consist of 6 pages (including references) and should follow
% the official style guidelines of the main conference.

\documentclass[11pt]{article}
\usepackage{acl2016}
\usepackage{times}
\usepackage{latexsym}
\usepackage{amssymb}
\usepackage{rotating}
\usepackage[hidelinks]{hyperref}

\aclfinalcopy % Uncomment this line for the final submission
%\def\aclpaperid{***} %  Enter the acl Paper ID here

\newcommand{\brot}[1]{\begin{turn}{70}\textbf{#1}\end{turn}}

\title{MeTA: A Modern C++ Data Sciences Toolkit} % keep "C++"?
\author{Sean Massung, Chase Geigle \and ChengXiang Zhai \\
            Computer Science Department, College of Engineering \\
            University of Illinois at Urbana-Champaign \\
            {\tt \{massung1,geigle1,czhai\}@illinois.edu}
}
\date{}

\def\meta/{\textsc{MeTA}}

\begin{document}

\maketitle

\begin{abstract}
\meta/ is developed to unite machine learning, information retrieval, and
natural language processing algorithms in one easy-to-use toolkit. Its focus on
indexing allows it to perform well on large datasets and support online
classification. \meta/'s liberal open source license encourages contributions,
and its extensive online documentation and tutorials make this process
straightforward. We perform experiments and show \meta/'s performance is
competitive with or better than existing software.
\end{abstract}

\section{A Unified Framework}

The existing environment of open source software solutions for text processing
is fragmented: there is rarely a single location for a wide variety of
algorithms. Tools tend to specialize on one particular area, and as such there
is a wide variety of tools one must sample when performing different data
science tasks. For text-mining tasks, this is even more apparent; it is
extremely difficult (if not impossible) to find tools that support both
traditional information retrieval tasks (like tokenization, indexing, and
search) alongside traditional machine learning tasks (like document
classification, regression, and topic modeling).

This places an undue burden on researchers---not only are they required to have
a detailed understanding of the research problem at hand, but they are now
forced to understand this fragmented nature of the open source software
community, find the appropriate software packages for their needs, and compile
and configure each appropriate tool. Even when this is all done, there is the
problem of data formatting---it is unlikely that the tools all have standardized
upon a single input format, so a certain amount of ``data munging'' is now
required. All of this detracts from the actual research task at hand, which has
a marked impact on the speed with which new ideas are discovered.
\todo{Make this less sales-pitch-y.}

The goal of the \meta/ project is to address these issues. In particular, we
provide a unifying framework for existing machine learning and natural language
processing algorithms, allowing researchers to quickly run controlled
experiments. We have modularized the feature generation, instance
representation, data storage formats, and algorithm implementations; this allows
for researchers, practitioners, and students to make seamless transitions along
any of these dimensions with minimal effort.

We have a liberal licensing scheme for \meta/. It is dual-licensed under the
University of Illinois/NCSA Open Source Licence and the MIT License. We use
these permissive licenses deliberately to allow for \meta/'s incorporation into
commercial products: we feel that the use of a copyleft license creates
unnecessary barriers between the code we develop as researchers and the
applications of our techniques to areas outside of the research community.
\todo{Reduce or discussion about commercial products.}

\begin{table*}[t]
    \begin{center}
    {\small
    \begin{tabular}{|l|c|c|c|c|c|c|c|c|}
        \hline
        & Indri & Lucene & MALLET & LIBLINEAR & SVM$^{MULT}$ & scikit &
        CoreNLP & \meta/ \\
        \hline
        Feature generation & & & & & & & & \checkmark \\
        Search & & & & & & & & \checkmark \\
        Classification & & & & & & & & \checkmark \\
        Regression & & & & & & & & \checkmark \\
        POS tagging & & & & & & & & \checkmark \\
        Parsing & & & & & & & & \checkmark \\
        Topic models & & & & & & & & \checkmark \\
        Language models & & & & & & & & \checkmark \\
        Word embeddings & & & & & & & & \checkmark \\
        Graph algorithms & & & & & & & & \checkmark \\
        Multithreading & & & & & & & & \checkmark \\
        \hline
    \end{tabular}
    \label{tab:feature-comp}
    \caption{Feature comparison of NLP, IR, and ML toolkits.}
    }
    \end{center}
\end{table*}

Table~\ref{tab:feature-comp} compares \meta/'s many features across various
toolkits. Due to space constraints, we only delve into the natural language
processing (NLP), information retrieval (IR), and machine learning (ML)
components in section~\ref{sec:experiments}. We briefly outline all components
here:

% might just convert to paragraphs with bolded "headers"
\begin{itemize}
    \item Feature generation
    \todo{Mention demos where available.}
    \item Search
    \item Classification
    \item Regression
    \item POS tagging
    \item Parsing
    \item Topic models
    \item Language models
    \item Word embeddings
    \item Graph algorithms
    \item Multithreading
\end{itemize}

\section{Built for Big Data}

Consistency across components is a key feature that allows \meta/ to work
well with large datasets. This is accomplished via a three-layer architecture.
On the first layer, we have tokenizers, analyzers, and all the text processing
that accompanies them. Once a document representation is determined, this tool
chain is run on a corpus. The indexes are the second layer; they provide an
efficient format for storing processed data. The third layer---the application
layer---interfaces solely with indexes. This means that we may use the same
index for running an SVM as we do to evaluate a ranking function, \emph{without
processing the data again}.

Since all applications use these indexes, \meta/ is able to support
out-of-core classification with some classifiers. We ran our large
classification dataset, rcv1~\cite{rcv1}, with limited memory (only 95MB) using
the \texttt{sgd} classifier. Where \textsc{liblinear} failed to run,
\meta/ was able to finish the classification in just under two minutes.

Besides using \meta/'s rich built-in feature generation, it is possible to
directly use \textsc{libsvm}-formatted data. This allows preprocessed datasets
to be run under \meta/'s algorithms. Additionally, \meta/'s
\texttt{forward\_index} (used for classification), is easily convertible to
\textsc{libsvm} format. Thus the reverse is also true: you may do feature
generation with \meta/, and use it to generate input for any other program that
supports \textsc{libsvm} format.

\section{Usability}

Consistency across components is a key feature that allows \meta/ to work
well with large datasets. This is accomplished via a three-layer architecture.
On the first layer, we have tokenizers, analyzers, and all the text processing
that accompanies them. Once a document representation is determined, this tool
chain is run on a corpus. The indexes are the second layer; they provide an
efficient format for storing processed data. The third layer---the application
layer---interfaces solely with indexes. This means that we may use the same
index for running an SVM as we do to evaluate a ranking function, \emph{without
processing the data again}.

Since all applications use these indexes, \meta/ is able to support out-of-core
classification with some classifiers. We ran our large classification dataset
that doesn't fit in memory---Webspam~\citep{Webb:2006:CEAS}---using the
\texttt{sgd} classifier. Where \textsc{liblinear} failed to run, \meta/ was able
to finish the classification in a few minutes.

Besides using \meta/'s rich built-in feature generation, it is possible to
directly use \textsc{libsvm}-formatted data. This allows preprocessed datasets
to be run under \meta/'s algorithms. Additionally, \meta/'s
\texttt{forward\_index} (used for classification), is easily convertible to
\textsc{libsvm} format. Thus the reverse is also true: you may do feature
generation with \meta/, and use it to generate input for any other program that
supports \textsc{libsvm} format.

\meta/ is hosted publicly on
GitHub\footnote{\url{https://github.com/meta-toolkit/meta/}}, which provides the
project with community involvement through its bug/issue tracker and fork/pull
request model. Its API is heavily
documented\footnote{\url{https://meta-toolkit.org/doxygen/namespaces.html}},
allowing the creation of Web-based applications (listed in
section~\ref{sec:framework}). The project website contains several tutorials
that cover the major aspects of the
toolkit\footnote{\url{https://meta-toolkit.org/}} to enable users to get started
as fast as possible with little friction. Additionally, a public
forum\footnote{\url{https://forum.meta-toolkit.org/}} is accessible for all
users to view and participate in user support topics, community-written
documentation, and developer discussions.

A major design point in \meta/ is to allow for most of the functionality to be
configured via a configuration file. This enables minimal effort exploratory
data analysis without having to write (or recompile) any code. Designing the
code in this way also encourages the components of the system to be pluggable:
the entire indexing process, for example, consists of several modular layers
which can be controlled by the configuration file.

An example snippet of a config file is given below; this creates a bigram
part-of-speech analyzer. Multiple \texttt{[[analyzers]]} sections may be added,
which \meta/ automatically combines while processing input.

{\small
\begin{verbatim}
   [[analyzers]]
   method = "ngram-pos"
   ngram = 2
   filter = [{type = "icu-tokenizer"},
             {type = "ptb-normalizer"}]
   crf-prefix = "crf/model/folder"
\end{verbatim}}

A simple class hierarchy allows users to add filters, analyzers, ranking
functions, and classifiers with full integration to the toolkit
(\emph{e.g.}\ one
may specify user-defined classes in the config file). The process for adding
these is detailed in the \meta/ online tutorials.

This low barrier of entry and ease in experiment setup has led \meta/ to be used
in text mining and analysis MOOCs reaching over 40,000
students\footnote{\url{https://www.coursera.org/course/textretrieval}}$^,$\footnote{\url{https://www.coursera.org/course/textanalytics}}.

Multi-language support is hard to do correctly. Many toolkits sidestep this
issue by only supporting ASCII text or the OS language; \meta/ supports multiple
(non-romance) languages by default, using the industry standard ICU
library\footnote{\url{http://site.icu-project.org/}}. This allows \meta/ to
tokenize arbitrarily-encoded text in many languages.

Unit tests ensure that contributors are confident that their modifications do
not break the toolkit. Unit tests are automatically run after each commit and
pull request, so developers immediately know if there is an issue (of course,
unit tests may be run manually before committing). The unit tests are run in a
continuous integration setup where \meta/ is compiled and run on Linux, Mac OS
X\footnote{\url{https://travis-ci.org/meta-toolkit/meta}}, and
Windows\footnote{\url{https://ci.appveyor.com/project/skystrife/meta}} under a
variety of compilers and software development configurations.

\section{Experiments}

\subsection{Natural Language Processing}

Talk about parsers and POS taggers.

Results in Fig~\ref{fig:all-nlp}.

\begin{figure*}[t]
\centering
{\small
\begin{tabular}{c@{\hskip 0.08in}c@{\hskip 0.08in}c}
\begin{tabular}{|l|r|r|}
        \hline & \textbf{Sentences} & \textbf{Examples} \\
        \hline
        Training & 42,685 & 969,905 \\
        Development & 6,526 & 148,158 \\
        Testing & 7,505 & 171,138 \\
        \hline
\end{tabular}
&
\begin{tabular}{|r|r|r|}
    \hline
    & \textbf{Token Accuracy} \\
    \hline
    Human annotators & 97.00\% \\
    CoreNLP & 97.32\% \\
    LTag-Spinal & 97.33\% \\
    SCCN & 97.50\% \\
    \meta/ & 97.03\% \\
    \hline
\end{tabular}
&
\begin{tabular}{|l|r|}
    \hline
    & \textbf{Token Accuracy} \\
    \hline
    CRFSuite & 96.0\% \\
    \meta/ & 95.2\% \\
    \hline
\end{tabular}

\\
& & \\
Penn Treebank WSJ\@.
&
POS-tagging.
&
Shallow parsing.
\\
\end{tabular}
}
\caption{Comparison of natural language processing systems.}
\label{fig:all-nlp}
\end{figure*}


\subsection{Information Retrieval}

\meta/'s IR performance is compared with two well-known search engine toolkits:
\textsc{Lucene}\footnotemark[12], a top-level Apache venture; and
\textsc{Lemur}~\cite{lemur}, a joint collaboration between the University of
Massachusetts Amherst and Carnegie Mellon University.

\footnotetext[12]{\url{http://lucene.apache.org/}}

First, we compare the speed at which the search engines can index each corpus
(Fig~\ref{fig:datasets}). We index unigram words with the identical
tokenization\footnotemark[1] across toolkits. Furthermore, all experiments were
run on a system with an Intel quad core i7-2760QM (2.40GHz) CPU, eight gigabytes
of memory, and a 7200 RPM disk. We used the datasets
20newsgroups\footnotemark[6], the blog authorship corpus\footnotemark[7], Reddit
comments\footnotemark[5], TREC 2006 Homepages\footnotemark[8], and English
Wikipedia\footnotemark[9].

\footnotetext[1]{%
All terms are lower-cased, stop words are removed from a common list of 431 stop
words, Porter2 (\meta/) or Porter (Lemur, Lucene) stemming is performed, a
maximum word length of 35 characters is set, and the original documents are not
stored in the index. In addition, each corpus is converted into a single file
with one document per line to reduce the effects of many file operations.
}

\footnotetext[5]{%
\url{http://www.reddit.com/r/datasets/comments/1mbsa2/155m_reddit_comments_over_15_days/}}
\footnotetext[6]{\url{http://qwone.com/~jason/20Newsgroups/}}
\footnotetext[7]{\url{http://ir.dcs.gla.ac.uk/test_collections/blogs06info.html}}
\footnotetext[8]{\url{http://trec.nist.gov/data.html}}
\footnotetext[9]{\url{http://en.wikipedia.org/wiki/Wikipedia:Database_download}}

We also compare how large the indexes are for each search engine
(Fig~\ref{fig:ir}). Ideally, since we are not storing the full text, the indexes
should be smaller than the original data, though this is not always the case.

To investigate retrieval speed, we created 500 queries for each dataset by
randomly selecting 500 documents, then randomly selecting one sentence from each
of the 500 documents. Results are also shown in Fig~\ref{fig:ir}. Relevance of
returned queries is not an interesting comparison since all indexes should be
storing identical information; that is, all tokenize the documents the same way
and a given retrieval function will perform the same operations on each indexed
document.

We see from the experimental results that \meta/ is a competitive information
retrieval framework. In indexing speed, it is between \textsc{Lemur} and
\textsc{Lucene}. Its index sizes are almost the smallest, except against
\textsc{Lucene}'s Homepages index. In terms of query speed, again \meta/ falls
between \textsc{Lemur} and \textsc{Lucene}.

\begin{figure*}[t]
\centering
{\small
\begin{tabular}{c@{\hskip 0.08in}c@{\hskip 0.08in}c}
\begin{tabular}{|l|r|r|r|}
        \hline & \textbf{Lemur} & \textbf{Lucene} & \textbf{MeTA} \\
        \hline
        \textbf{20news} & 4s        & 4s     & 5s \\
        \textbf{Blog}   & 1m 22s    & 43s    & 1m 2s \\
        \textbf{Reddit} & 15m 41s   & 4m 2s  & 10m 35s \\
        \textbf{Homes}  & 15m 31s   & 7m 11s & 13m 20s \\
        \textbf{Wiki}   & 1h 39m 7s & 33m 0s & 1h 21m 14s \\
        \hline
\end{tabular}
&
\begin{tabular}{|r|r|r|}
    \hline \textbf{Lemur} & \textbf{Lucene} & \textbf{MeTA} \\
    \hline
     62 MB  & 8.5 MB & 7.1 MB \\
     775 MB & 119 MB & 40 MB  \\
     6.5 GB & 636 MB & 1.1 GB \\
     7.1 GB & 1.3 GB & 479 MB \\
     30 GB  & 5.6 GB & 2.5 GB \\
    \hline
\end{tabular}
&
\begin{tabular}{|r|r|r|}
    \hline \textbf{Lemur} & \textbf{Lucene} & \textbf{MeTA} \\
    \hline
     37s        & 2s  & 2s \\
     1m 36s     & 4s  & 7s \\
     1h 50m 55s & 29s & 18m 28s \\
     11m 12s    & 39s & 1m 54s \\
     1h 10m 4s  & 2m 20s & 14m 21s \\
    \hline
\end{tabular}

\\
& & \\
Indexing speed.
&
Size of each index.
&
Speed for 500 queries.
\\
\end{tabular}
}
\caption{Comparison of information retrieval systems.}
\label{fig:ir}
\end{figure*}


\subsection{Machine Learning}

\meta/'s ML performance is compared with \textsc{liblinear}~\cite{liblinear}, a
well-known SVM library; \textsc{scikit-learn}~\cite{scikit} a Python ML library;
and \textsc{SVMMulticlass}~\cite{svmmulticlass}, a competitor to
\textsc{liblinear}. Statistics for datasets used in both parts can be found in
Fig~\ref{fig:ir}.

For the machine learning evaluation, we focus on linear classification across
toolkits. Many applications in \meta/ are in a textual domain, and linear
classification lends itself to the high-dimensional space that comes with text
documents. All experiments are performed on a system with a dual core Intel Core
i5 CPU (M460) clocked at 2.53GHz and eight gigabytes of RAM.

Four datasets (20news, Blog, Newegg, and the Yelp Academic
Dataset\footnotemark[10]) are textual datasets. The Newegg (crawled ourselves)
and Yelp datasets are review datasets, and we consider only a partial list of
the whole dataset where we have removed all three-star reviews. We used MeTA for
tokenization and feature generation, applying the same constraints as we did for
the indexing tests. Randomized training (two thirds) and test splits (one third)
were generated for each of these datasets. For rcv1, we used the existing
tokenization and training/test splits available on the LIBSVM data
website~\footnotemark[11].

\footnotetext[10]{\url{https://www.yelp.com/academic_dataset}}
\footnotetext[11]{\url{http://www.csie.ntu.edu.tw/~cjlin/libsvmtools/datasets/}}

In Fig~\ref{fig:classification}, we can see that \meta/ performs well both in
terms of speed and accuracy, and presents itself as a viable option in the
machine learning domain.

\begin{figure}
\centering
{\small
\begin{tabular}{|l|r|r|r|r|}
    \hline & \textbf{Docs} & \textbf{Size} & \textbf{$|D|_{avg}$} & \textbf{$|V|$} \\
    \hline
    \textbf{20news}  & 18,828     & 39 MB  & 223.4      & 97,667     \\
    \textbf{Blog}    & 18,713     & 654 MB & 3432.1     & 462,425    \\
    \textbf{Homes}   & 324,553    & 5.0 GB & 1742.3     & 3,609,811  \\
    \textbf{Newegg}  & 106,998    & 57 MB  & 55.6       & 38,533     \\
    \textbf{rcv1}    & 697,641    & 1.2 GB & \emph{N/A} & 47,236 \\
    \textbf{Reddit}  & 15,497,530 & 2.6 GB & 18.6       & 1,942,065  \\
    \textbf{Yelp}    & 194,543    & 130 MB & 74.9       & 75,703 \\
    \textbf{Wiki}    & 8,284,146  & 25 GB  & 272.3      & 15,248,289 \\
    \hline
\end{tabular}
}
\caption{Datasets for IR and ML experiments.}
\label{fig:datasets}
\end{figure}

\begin{figure}[t]
\centering
{\small
\begin{tabular}{|l|r|r|r|r|}
  \hline
  & \textbf{liblinear} & \textbf{scikit} &
  \textbf{SVM\textsuperscript{mult}} & \textbf{MeTA}\\\hline
  \textbf{20news}                    & 88.18\%          & 78.03\%  & 70.27\%          & \textbf{89.10\%}\\
  $(k=20)$                           & 15.64s           & 5.28s    & 35.36s           & 7.55s\\\hline
  \textbf{Blog}                      & 45.53\%          & 35.91\%  & \textbf{46.22\%} & 41.14\%\\
  $(k=7)$                            & 310.6s           & 46.94s   & 164.6s           & 28.34s\\\hline
  \textbf{Newegg}                    & 90.39\%          & 90.71\%  & 78.71\%          & \textbf{91.96\%}\\
  $(k=2)$                            & 23.90s           & 10.93s   & 3.806s           & 4.182s\\\hline
  \textbf{Yelp}                      & 91.65\%          & 92.02\%  & 80.22\%          & \textbf{92.98\%}\\
  $(k=2)$                            & 51.89s           & 27.48s   & 9.700s           & 8.854s\\\hline
  \textbf{rcv1}                      & \textbf{88.43\%} & 87.68\%  & 55.21\%          & 79.06\%\\
  $(k=2)$                            & 21.04s           & 99.07s   & 3.37s            & 37.42s\\\hline
\end{tabular}
}
\caption{Accuracy and speed classification results.}
\label{fig:classification}
\end{figure}


\section{Demo}

To do? Show pics? Give API?

\section{Conclusions}

\meta/ is a valuable resource for text mining applications; it is a viable and
competitive alternative to existing toolkits that unifies algorithms from NLP,
IR, and ML\@. \meta/ is an extensible, consistent framework that enables quick
development of complex application systems.


%\small
\bibliography{bib}
\bibliographystyle{acl2016}
%\normalsize

\end{document}
