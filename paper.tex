% Submissions may consist of 6 pages (including references) and should follow
% the official style guidelines of the main conference.

\documentclass[11pt]{article}
\usepackage{acl2016}
\usepackage{times}
\usepackage{latexsym}
\usepackage{amssymb}
\usepackage{rotating}
\usepackage[hidelinks]{hyperref}

\aclfinalcopy % Uncomment this line for the final submission
%\def\aclpaperid{***} %  Enter the acl Paper ID here

\newcommand{\brot}[1]{\begin{turn}{70}\textbf{#1}\end{turn}}

\title{MeTA: A Modern C++ Data Sciences Toolkit} % keep "C++"?
\author{Sean Massung, Chase Geigle \and ChengXiang Zhai \\
            Computer Science Department, College of Engineering \\
            University of Illinois at Urbana-Champaign \\
            {\tt \{massung1,geigle1,czhai\}@illinois.edu}
}
\date{}

\def\meta/{\textsc{MeTA}}

\begin{document}

\maketitle

\begin{abstract}
\meta/ is developed to unite machine learning, information retrieval, and
natural language processing algorithms in one easy-to-use toolkit. Its focus on
indexing allows it to perform well on large datasets and support online
classification. \meta/'s liberal open source license encourages contributions,
and its extensive online documentation and tutorials make this process
straightforward. We perform experiments and show \meta/'s performance is
competitive with or better than existing software.
\end{abstract}

\section{A Unified Framework}
\label{sec:framework}

As NLP techniques become more and more mature, we have great opportunities to
use them to develop and support many applications, such as search engines,
classifiers, and integrative applications that involve multiple components. It's
possible to develop each application from scratch, but it's much more efficient
to have a general toolkit that supports multiple application types.

Existing tools tend to specialize on one particular area, and as such there is a
wide variety of tools one must sample when performing different data science
tasks. For text-mining tasks, this is even more apparent; it is extremely
difficult (if not impossible) to find tools that support both traditional
information retrieval tasks (like tokenization, indexing, and search) alongside
traditional machine learning tasks (like document classification, regression,
and topic modeling).

Table~\ref{tab:feature-comp} compares \meta/'s many features across various
dimensions. Note that only \meta/ % can find the peers
satisfies all the areas while other toolkits focus on a particular area. In the
case where the desired functionality is scattered, data science students,
researchers, and practitioners must find the appropriate software packages for
their needs and compile and configure each appropriate tool. Then, there is the
problem of data formatting---it is unlikely that the tools all have standardized
upon a single input format, so a certain amount of ``data munging'' is required.
All of this detracts from the actual task at hand, which has a marked impact on
productivity.

The goal of the \meta/ project is to address these issues. In particular, we
provide a unifying framework for existing machine learning and natural language
processing algorithms, allowing researchers to quickly run controlled
experiments. We have modularized the feature generation, instance
representation, data storage formats, and algorithm implementations; this allows
users to make seamless transitions along any of these dimensions with minimal
effort. Finally, \meta/ is dual-licensed under the University of Illinois/NCSA
Open Source Licence and the MIT License to reach the broadest audience possible.

\begin{table*}[t]
    \begin{center}
    {\small
    \begin{tabular}{|l|c|c|c|c|c|c|c|c|}
        \hline
        & \textbf{Indri} & \textbf{Lucene} & \textbf{MALLET} &
        \textbf{LIBLINEAR} & \textbf{SVM$^{MULT}$} & \textbf{scikit} &
        \textbf{CoreNLP} & \textbf{\meta/} \\
        & \emph{IR} & \emph{IR} & \emph{ML/NLP} & \emph{ML} & \emph{ML} &
        \emph{ML/NLP} & \emph{ML/NLP} & \emph{all} \\
        \hline
        Feature generation & \checkmark & \checkmark & \checkmark & & &
        \checkmark & \checkmark & \checkmark \\
        Search & \checkmark & \checkmark & & & & & & \checkmark \\
        Classification & & & \checkmark & \checkmark & \checkmark & \checkmark &
        \checkmark & \checkmark \\
        Regression & & & \checkmark & \checkmark & \checkmark & \checkmark &
        \checkmark & \checkmark \\
        POS tagging & & & \checkmark & & & & \checkmark & \checkmark \\
        Parsing & & & & & & & \checkmark & \checkmark \\
        Topic models & & & \checkmark & & & \checkmark & & \checkmark \\
        $n$-gram LM & & & & & & & & \checkmark \\
        Word embeddings & & & \checkmark & & & & \checkmark & \checkmark \\
        Graph algorithms & & & & & & & & \checkmark \\
        Multithreading & & \checkmark & \checkmark & & & \checkmark & \checkmark
        & \checkmark \\
        \hline
    \end{tabular}
    \caption{Toolkit feature comparison. Citations for all toolkits may be found
        in their respective comparison sections.}
    \label{tab:feature-comp}
    }
    \end{center}
\end{table*}

Due to space constraints, in this paper, we only delve into \meta/'s natural
language processing (NLP), information retrieval (IR), and machine learning (ML)
components in section~\ref{sec:experiments}. We briefly outline all components
here:

\textbf{Feature generation}.
\meta/ has a collection of tokenizers, filters, and analyzers that convert raw
text into a feature representation. Basic features are $n$-gram words, but other
analyzers make use of different parts of the toolkit, such as POS tag $n$-grams
and parse tree features. An arbitrary number of feature representations may be
combined; for example, a document could be represented as unigram words, bigram
POS tags, and parse tree rewrite rules. Users can easily add their own feature
types as well, such as sentence length distribution in a document.

\textbf{Search}.
The \meta/ search engine can store document feature vectors in an inverted index
and score them with respect to a query. Rankers include vector space models such
as Okapi BM25~\citep{bm25} and probabilistic models like Dirichlet prior
smoothing~\citep{zhai-lm}. A search demo is
online\footnote{\url{https://meta-toolkit.org/search-demo.html}}.

\textbf{Classification}.
\meta/ includes a stochastic gradient descent (SGD) implementation with
pluggable loss functions, allowing creation of an SVM classifier (among others).
All-\emph{vs}-all and one-\emph{vs}-all ensemble methods for binary classifiers
allow multiclass classification. Other classifiers like na{\"i}ve Bayes and
$k$-nearest neighbors also exist. A confusion matrix class and significance
testing framework allow evaluation and comparison of different methods and
feature representations.

\textbf{Regression}.
Regression via SGD predicts real-valued responses from featurized documents.
Evaluation metrics such as mean squared error and $R^2$ score allow model
comparison.

\textbf{POS tagging}.
\meta/ contains a linear-chain conditional random field for POS tagging and
chunking applications, learned using $\ell_2$ regularized SGD~\citep{crf}. It
also contains an efficient greedy averaged perceptron tagger~\citep{greedy}.

\textbf{Parsing}.
A fast shift-reduce constituency parser using generalized averaged perceptron
~\citep{const-parsing} is \meta/'s grammatical parser. Parse tree featurizers
implement different types of structural tree representations~\citep{structural}.
An NLP demo online presents tokenization, POS-tagging, and
parsing\footnote{\url{https://meta-toolkit.org/nlp-demo.html}}.

\textbf{Topic models}.
\meta/ can learn topic models over any feature representation using collapsed
variational Bayes~\citep{cvb}, collapsed Gibbs sampling~\citep{gibbs}, stochastic
collapsed variational Bayes~\citep{scvb}, or approximate distributed
LDA~\citep{pargibbs}.

\textbf{$n$-gram language models} (LMs).
\meta/ takes an ARPA-formatted
input\footnote{\url{http://www.speech.sri.com/projects/srilm/manpages/ngram-format.5.html}}
and creates a language model that can be queried for token sequence
probabilities or used in downstream applications like
SyntacticDiff~\citep{syndiff}.
% MPH not merged yet, mention?

\textbf{Word embeddings}.
The GloVe algorithm~\citep{glove} is implemented in a streaming framework and
also features an interactive semantic relationship demo. Word vectors can be
used in other applications as part of the \meta/ API\@.

\textbf{Graph algorithms}.
Directed and undirected graph implementations exist and various algorithms such
as betweenness centrality, PageRank, and myopic search are available. Random
graph generation models like Watts-Strogatz and preferential attachment exist.
For these algorithms see \citet{networks}.

\textbf{Multithreading}.
When possible, all \meta/ algorithms and applications are parallelized using C++
threads to make full use of available resources.

\section{Built for Big Data}

Consistency across components is a key feature that allows \meta/ to work
well with large datasets. This is accomplished via a three-layer architecture.
On the first layer, we have tokenizers, analyzers, and all the text processing
that accompanies them. Once a document representation is determined, this tool
chain is run on a corpus. The indexes are the second layer; they provide an
efficient format for storing processed data. The third layer---the application
layer---interfaces solely with indexes. This means that we may use the same
index for running an SVM as we do to evaluate a ranking function, \emph{without
processing the data again}.

Since all applications use these indexes, \meta/ is able to support
out-of-core classification with some classifiers. We ran our large
classification dataset, rcv1~\cite{rcv1}, with limited memory (only 95MB) using
the \texttt{sgd} classifier. Where \textsc{liblinear} failed to run,
\meta/ was able to finish the classification in just under two minutes.

Besides using \meta/'s rich built-in feature generation, it is possible to
directly use \textsc{libsvm}-formatted data. This allows preprocessed datasets
to be run under \meta/'s algorithms. Additionally, \meta/'s
\texttt{forward\_index} (used for classification), is stored as \textsc{libsvm}
format. Thus the reverse is also true: you may do feature generation with
\meta/, and use its index as input to any other program that supports
\textsc{libsvm} format.

\section{Usability}

Consistency across components is a key feature that allows \meta/ to work
well with large datasets. This is accomplished via a three-layer architecture.
On the first layer, we have tokenizers, analyzers, and all the text processing
that accompanies them. Once a document representation is determined, this tool
chain is run on a corpus. The indexes are the second layer; they provide an
efficient format for storing processed data. The third layer---the application
layer---interfaces solely with indexes. This means that we may use the same
index for running an SVM as we do to evaluate a ranking function, \emph{without
processing the data again}.

Since all applications use these indexes, \meta/ is able to support out-of-core
classification with some classifiers. We ran our large classification dataset
that doesn't fit in memory---Webspam~\citep{Webb:2006:CEAS}---using the
\texttt{sgd} classifier. Where \textsc{liblinear} failed to run, \meta/ was able
to finish the classification in a few minutes.

Besides using \meta/'s rich built-in feature generation, it is possible to
directly use \textsc{libsvm}-formatted data. This allows preprocessed datasets
to be run under \meta/'s algorithms. Additionally, \meta/'s
\texttt{forward\_index} (used for classification), is easily convertible to
\textsc{libsvm} format. Thus the reverse is also true: you may do feature
generation with \meta/, and use it to generate input for any other program that
supports \textsc{libsvm} format.

\meta/ is hosted publicly on
GitHub\footnote{\url{https://github.com/meta-toolkit/meta/}}, which provides the
project with community involvement through its bug/issue tracker and fork/pull
request model. Its API is heavily
documented\footnote{\url{https://meta-toolkit.org/doxygen/namespaces.html}},
allowing the creation of Web-based applications (listed in
section~\ref{sec:framework}). The project website contains several tutorials
that cover the major aspects of the
toolkit\footnote{\url{https://meta-toolkit.org/}} to enable users to get started
as fast as possible with little friction. Additionally, a public
forum\footnote{\url{https://forum.meta-toolkit.org/}} is accessible for all
users to view and participate in user support topics, community-written
documentation, and developer discussions.

A major design point in \meta/ is to allow for most of the functionality to be
configured via a configuration file. This enables minimal effort exploratory
data analysis without having to write (or recompile) any code. Designing the
code in this way also encourages the components of the system to be pluggable:
the entire indexing process, for example, consists of several modular layers
which can be controlled by the configuration file.

An example snippet of a config file is given below; this creates a bigram
part-of-speech analyzer. Multiple \texttt{[[analyzers]]} sections may be added,
which \meta/ automatically combines while processing input.

{\small
\begin{verbatim}
   [[analyzers]]
   method = "ngram-pos"
   ngram = 2
   filter = [{type = "icu-tokenizer"},
             {type = "ptb-normalizer"}]
   crf-prefix = "crf/model/folder"
\end{verbatim}}

A simple class hierarchy allows users to add filters, analyzers, ranking
functions, and classifiers with full integration to the toolkit
(\emph{e.g.}\ one
may specify user-defined classes in the config file). The process for adding
these is detailed in the \meta/ online tutorials.

This low barrier of entry and ease in experiment setup has led \meta/ to be used
in text mining and analysis MOOCs reaching over 40,000
students\footnote{\url{https://www.coursera.org/course/textretrieval}}$^,$\footnote{\url{https://www.coursera.org/course/textanalytics}}.

Multi-language support is hard to do correctly. Many toolkits sidestep this
issue by only supporting ASCII text or the OS language; \meta/ supports multiple
(non-romance) languages by default, using the industry standard ICU
library\footnote{\url{http://site.icu-project.org/}}. This allows \meta/ to
tokenize arbitrarily-encoded text in many languages.

Unit tests ensure that contributors are confident that their modifications do
not break the toolkit. Unit tests are automatically run after each commit and
pull request, so developers immediately know if there is an issue (of course,
unit tests may be run manually before committing). The unit tests are run in a
continuous integration setup where \meta/ is compiled and run on Linux, Mac OS
X\footnote{\url{https://travis-ci.org/meta-toolkit/meta}}, and
Windows\footnote{\url{https://ci.appveyor.com/project/skystrife/meta}} under a
variety of compilers and software development configurations.

\section{Experiments}
\label{sec:experiments}

To demonstrate \meta/'s effectiveness in three crucial fields, we evaluate its
performance in NLP, IR, and ML tasks. All experiments were performed on a
workstation with an Intel(R) Core(TM) i7-5820K CPU, 16 GB of RAM, and a 4
TB 5900 RPM disk.

\subsection{Natural Language Processing}

Talk about parsers and POS taggers.

Results in Table~\ref{table:nlp-pos}.
Results in Table~\ref{table:nlp-shallow}.

\begin{table}[t]
\centering
{\small
\begin{tabular}{|l|r|r|}
        \hline & \textbf{Sentences} & \textbf{Examples} \\
        \hline
        Training & 42,685 & 969,905 \\
        Development & 6,526 & 148,158 \\
        Testing & 7,505 & 171,138 \\
        \hline
\end{tabular}
}
\caption{NLP: the Penn Treebank dataset.}
\label{table:nlp-datasets}
\end{table}

\begin{table}[t]
\centering
{\small
\begin{tabular}{|r|r|r|}
    \hline
    & \textbf{Token Accuracy} \\
    \hline
    Human annotators & 97.00\% \\
    CoreNLP & 97.32\% \\
    LTag-Spinal & 97.33\% \\
    SCCN & 97.50\% \\
    \meta/ & 97.03\% \\
    \hline
\end{tabular}
}
\caption{(NLP) Part-of-speech tagging.}
\label{table:nlp-pos}
\end{table}

\begin{table}[t]
\centering
{\small
\begin{tabular}{|l|r|}
    \hline
    & \textbf{Token Accuracy} \\
    \hline
    ??? & 8000\% \\
    \meta/ & 9000\% \\
    \hline
\end{tabular}
}
\caption{(NLP) Grammatical parsing.}
\label{table:nlp-shallow}
\end{table}


\subsection{Information Retrieval}

\meta/'s IR performance is compared with two well-known search engine toolkits:
\textsc{Lucene}'s latest version 5.5.0\footnote{\url{http://lucene.apache.org/}} and
\textsc{Indri}'s version 5.9\footnote{Indri 5.10 does not provide source
    code packages and thus could not be used. It is also known as
\textsc{Lemur}.}~\cite{lemur}.

For the IR experiments, we use the TREC blog06~\cite{blog06} permalink documents
and TREC gov2 corpus~\cite{gov2}. To ensure a more uniform indexing environment,
all HTML is cleaned before indexing. In addition, each corpus is converted into
a single file with one document per line to reduce the effects of many file
operations.

During indexing, terms are lower-cased, stop words are removed from a common
list of 431 stop words, Porter2 (\meta/) or Porter (Indri, Lucene) stemming is
performed, a maximum word length of 32 characters is set, original documents are
not stored in the index, and term position information is not
stored\footnote{For Indri, we are unable to disable positions information
storage.}.

We compare the following: indexing speed (Table~\ref{table:ir-indexing}), index
size in Table~\ref{table:ir-index-size}, query speed in
Table~\ref{table:ir-query-speed}, and query accuracy with BM25 ($k_1=0.9,
b=0.4$) in Table~\ref{table:ir-map}. We use the standard TREC queries associated
with each dataset and score each system's search results with the usual
\texttt{trec\_eval} program\footnote{\url{http://trec.nist.gov/trec_eval/}}.

\meta/ leads in indexing speed, though we note that \meta/'s default
indexer is multithreaded and \textsc{Lucene} does not provide a parallel
one\footnote{Additionally, we did not feel that writing a correct and threadsafe
indexer as a user is something to be reasonably expected.}. \meta/ creates the
smallest index for gov2 while \textsc{Lucene} creates the smallest index for
blog06; \textsc{Indri} greatly lags behind both. \meta/ follows \textsc{Lucene}
closely in retrieval speed, with \textsc{Indri} again lagging. As expected,
query performance between the three systems is relatively even, and we attribute
any small difference in MAP or precision to idiosyncrasies during tokenization.

\begin{table}
\centering
{\small
\begin{tabular}{|l|r|r|r|r|}
    \hline & \textbf{Docs} & \textbf{Size} \emph{(original)} & \textbf{$|D|_{avg}$} &
    \textbf{$|V|$} \\
    \hline
    \textbf{Blog06} & 3,215,171 & 26 GB \emph{(89)} & 782.3 & 10,971,746 \\
    \textbf{Gov2} & 25,205,179 & 147 GB \emph{(426)} & & \\
    \hline
\end{tabular}
\caption{(IR) The two TREC datasets used. Index size in parenthesis is the
original size before HTML cleaning.}
\label{table:ir-datasets}
}
\end{table}

\begin{table}[t]
\centering
{\small
\begin{tabular}{|l|r|r|r|}
        \hline & \textbf{Indri} & \textbf{Lucene} & \textbf{MeTA} \\
        \hline
        \textbf{Blog06} & & & \\
        \textbf{Gov2}    & & & \\
        \hline
\end{tabular}
\caption{IR: indexing speed.}
}
\label{table:ir-indexing}
\end{table}

\begin{figure}[t]
\centering
{\small
\begin{tabular}{|l|r|r|r|}
    \hline & \textbf{Indri} & \textbf{Lucene} & \textbf{MeTA} \\
    \hline
    \textbf{Blog06} & & & \\
    \textbf{Gov2} & & & \\
    \hline
\end{tabular}
\caption{IR: index size.}
}
\label{fig:ir-index-size}
\end{figure}

\begin{table}[t]
\centering
{\small
\begin{tabular}{|l|r|r|r|}
    \hline & \textbf{Indri} & \textbf{Lucene} & \textbf{MeTA} \\
    \hline
    \textbf{Blog06} & 55.0s & 1.6s & 43.6s \\
    \textbf{Gov2} & & 57.53s & 14m 54s \\
    \hline
\end{tabular}
\caption{(IR) Query speed.}
\label{table:ir-query-speed}
}
\end{table}

\begin{figure}[t]
\centering
{\small
\begin{tabular}{|l|r|r|r|}
    \hline & \textbf{Lemur} & \textbf{Lucene} & \textbf{MeTA} \\
    \hline
    \textbf{Blogs06} & & & \\
    \textbf{Gov2} & & & \\
    \hline
\end{tabular}
}
\caption{IR: query performance via Mean Average Precision.}
\label{fig:ir-map}
\end{figure}


\subsection{Machine Learning}

\meta/'s ML performance is compared with \textsc{liblinear}~\cite{liblinear}, a
well-known SVM library; \textsc{scikit-learn}~\cite{scikit} a Python ML library;
and \textsc{SVMMulticlass}~\cite{svmmulticlass}, a competitor to
\textsc{liblinear}. We focus on linear classification across these tools.
Statistics for the five datasets used can be found in
Table~\ref{table:ml-datasets}.

Look at MALLET~\cite{mallet}.

Four of the datasets
(20news\footnote{\url{http://qwone.com/~jason/20Newsgroups/}}, Newegg, and the
Yelp Academic Dataset\footnote{\url{https://www.yelp.com/academic_dataset}}) are
textual datasets. The Newegg (crawled ourselves) and Yelp datasets are review
datasets, and we consider only a partial list of the whole dataset where we have
removed all three-star reviews. We used MeTA for tokenization and feature
generation, applying the same constraints as we did for the indexing tests.
Randomized training (two thirds) and test splits (one third) were generated for
each of these datasets. For rcv1, we used the existing tokenization and
training/test splits available on the LIBSVM data
website~\footnote{\url{http://www.csie.ntu.edu.tw/~cjlin/libsvmtools/datasets/}}.

In Table~\ref{table:ml-exp}, we can see that \meta/ performs well both in
terms of speed and accuracy, and presents itself as a viable option in the
machine learning domain.

\begin{table}[t]
\centering
{\small
\begin{tabular}{|l|r|r|r|r|}
    \hline
    & \multicolumn{1}{c|}{\textbf{Docs}}
    & \multicolumn{1}{c|}{\textbf{Size}}
    & \multicolumn{1}{c|}{\textbf{Classes}}
    & \multicolumn{1}{c|}{\textbf{Features}}\\
    \hline
    \textbf{20news}  & 18,846     & 86 MB  & 20 & 112,377 \\
    \textbf{Blog}    & 19,320     & 778 MB & 3  & 548,812 \\
    \textbf{rcv1}    & 804,414    & 1.1 GB & 2  & 47,152 \\
    \textbf{Webspam} & 350,000    & 24 GB  & 2  & 16,609,143 \\
    \hline
\end{tabular}
}
\caption{(ML) Datasets used.}
\label{table:ml-datasets}
\end{table}

\begin{table}[t]
\centering
{\small
\begin{tabular}{|l|r|r|r|r|}
  \hline
  & \textbf{liblinear} & \textbf{scikit} &
  \textbf{SVM\textsuperscript{mult}} & \textbf{\meta/}\\\hline
  \multirow{2}{*}{\textbf{20news}}   & 79.4\% & 74.3\% & 67.1\% & 80.1\% \\
                                     & 2.58s  & 0.326s & 2.54s  & 0.648s \\\hline
  \multirow{2}{*}{\textbf{Blog}}     & 75.8\% & 76.2\% & 72.2\% & 72.2\% \\
                                     & 61.3s  & 0.801s & 17.5s  & 1.11s \\\hline
  \multirow{2}{*}{\textbf{rcv1}}     & 94.7\% & 94.0\% & 83.6\% & 94.8\% \\
                                     & 17.6s  & 1.66s  & 2.01s  & 3.44s \\\hline
  \multirow{2}{*}{\textbf{Webspam}}
  & \multirow{2}{*}{\xmark}
  & 97.4\%
  & \multirow{2}{*}{\xmark}
  & 99.4\% \\
                                     & & 11m 52s& & 1m 16s \\\hline
\end{tabular}
}
\caption{(ML) Accuracy and speed classification results. Reported time is
  to both train and test the model. For all except Webspam, this excludes
  IO.}
\label{table:ml-exp}
\end{table}


\section{Demo}

To do? Show pics? Give API?

Easy to use in Web app or as an API\@.

\section{Conclusion}

\meta/ is a valuable resource for text mining applications; it is a viable and
competitive alternative to existing toolkits that unifies algorithms from NLP,
IR, and ML\@. \meta/ is an extensible, consistent framework that enables quick
development of complex application systems.


%\small
\bibliography{bib}
\bibliographystyle{acl2016}
%\normalsize

\end{document}
