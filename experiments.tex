\section{Experiments}

\subsection{Natural Language Processing}

Talk about parsers and POS taggers.

Results in Fig~\ref{fig:all-nlp}.

\begin{figure*}[t]
\centering
{\small
\begin{tabular}{c@{\hskip 0.08in}c@{\hskip 0.08in}c}
\begin{tabular}{|l|r|r|}
        \hline & \textbf{Sentences} & \textbf{Examples} \\
        \hline
        Training & 42,685 & 969,905 \\
        Development & 6,526 & 148,158 \\
        Testing & 7,505 & 171,138 \\
        \hline
\end{tabular}
&
\begin{tabular}{|r|r|r|}
    \hline
    & \textbf{Token Accuracy} \\
    \hline
    Human annotators & 97.00\% \\
    CoreNLP & 97.32\% \\
    LTag-Spinal & 97.33\% \\
    SCCN & 97.50\% \\
    \meta/ & 97.03\% \\
    \hline
\end{tabular}
&
\begin{tabular}{|l|r|}
    \hline
    & \textbf{Token Accuracy} \\
    \hline
    CRFSuite & 96.0\% \\
    \meta/ & 95.2\% \\
    \hline
\end{tabular}

\\
& & \\
Penn Treebank WSJ\@.
&
POS-tagging.
&
Shallow parsing.
\\
\end{tabular}
}
\caption{Comparison of natural language processing systems.}
\label{fig:all-nlp}
\end{figure*}


\subsection{Information Retrieval}

\meta/'s IR performance is compared with two well-known search engine toolkits:
\textsc{Lucene}\footnotemark[12], a top-level Apache venture; and
\textsc{Lemur}~\cite{lemur}, a joint collaboration between the University of
Massachusetts Amherst and Carnegie Mellon University.

\footnotetext[12]{\url{http://lucene.apache.org/}}

First, we compare the speed at which the search engines can index each corpus
(Fig~\ref{fig:datasets}). We index unigram words with the identical
tokenization\footnotemark[1] across toolkits. Furthermore, all experiments were
run on a system with an Intel quad core i7-2760QM (2.40GHz) CPU, eight gigabytes
of memory, and a 7200 RPM disk. We used the datasets
20newsgroups\footnotemark[6], the blog authorship corpus\footnotemark[7], Reddit
comments\footnotemark[5], TREC 2006 Homepages\footnotemark[8], and English
Wikipedia\footnotemark[9].

\footnotetext[1]{%
All terms are lower-cased, stop words are removed from a common list of 431 stop
words, Porter2 (\meta/) or Porter (Lemur, Lucene) stemming is performed, a
maximum word length of 35 characters is set, and the original documents are not
stored in the index. In addition, each corpus is converted into a single file
with one document per line to reduce the effects of many file operations.
}

\footnotetext[5]{%
\url{http://www.reddit.com/r/datasets/comments/1mbsa2/155m_reddit_comments_over_15_days/}}
\footnotetext[6]{\url{http://qwone.com/~jason/20Newsgroups/}}
\footnotetext[7]{\url{http://ir.dcs.gla.ac.uk/test_collections/blogs06info.html}}
\footnotetext[8]{\url{http://trec.nist.gov/data.html}}
\footnotetext[9]{\url{http://en.wikipedia.org/wiki/Wikipedia:Database_download}}

We also compare how large the indexes are for each search engine
(Fig~\ref{fig:ir}). Ideally, since we are not storing the full text, the indexes
should be smaller than the original data, though this is not always the case.

To investigate retrieval speed, we created 500 queries for each dataset by
randomly selecting 500 documents, then randomly selecting one sentence from each
of the 500 documents. Results are also shown in Fig~\ref{fig:ir}. Relevance of
returned queries is not an interesting comparison since all indexes should be
storing identical information; that is, all tokenize the documents the same way
and a given retrieval function will perform the same operations on each indexed
document.

We see from the experimental results that \meta/ is a competitive information
retrieval framework. In indexing speed, it is between \textsc{Lemur} and
\textsc{Lucene}. Its index sizes are almost the smallest, except against
\textsc{Lucene}'s Homepages index. In terms of query speed, again \meta/ falls
between \textsc{Lemur} and \textsc{Lucene}.

\begin{figure*}[t]
\centering
{\small
\begin{tabular}{c@{\hskip 0.08in}c@{\hskip 0.08in}c}
\begin{tabular}{|l|r|r|r|}
        \hline & \textbf{Lemur} & \textbf{Lucene} & \textbf{MeTA} \\
        \hline
        \textbf{20news} & 4s        & 4s     & 5s \\
        \textbf{Blog}   & 1m 22s    & 43s    & 1m 2s \\
        \textbf{Reddit} & 15m 41s   & 4m 2s  & 10m 35s \\
        \textbf{Homes}  & 15m 31s   & 7m 11s & 13m 20s \\
        \textbf{Wiki}   & 1h 39m 7s & 33m 0s & 1h 21m 14s \\
        \hline
\end{tabular}
&
\begin{tabular}{|r|r|r|}
    \hline \textbf{Lemur} & \textbf{Lucene} & \textbf{MeTA} \\
    \hline
     62 MB  & 8.5 MB & 7.1 MB \\
     775 MB & 119 MB & 40 MB  \\
     6.5 GB & 636 MB & 1.1 GB \\
     7.1 GB & 1.3 GB & 479 MB \\
     30 GB  & 5.6 GB & 2.5 GB \\
    \hline
\end{tabular}
&
\begin{tabular}{|r|r|r|}
    \hline \textbf{Lemur} & \textbf{Lucene} & \textbf{MeTA} \\
    \hline
     37s        & 2s  & 2s \\
     1m 36s     & 4s  & 7s \\
     1h 50m 55s & 29s & 18m 28s \\
     11m 12s    & 39s & 1m 54s \\
     1h 10m 4s  & 2m 20s & 14m 21s \\
    \hline
\end{tabular}

\\
& & \\
Indexing speed.
&
Size of each index.
&
Speed for 500 queries.
\\
\end{tabular}
}
\caption{Comparison of information retrieval systems.}
\label{fig:ir}
\end{figure*}


\subsection{Machine Learning}

\meta/'s ML performance is compared with \textsc{liblinear}~\cite{liblinear}, a
well-known SVM library; \textsc{scikit-learn}~\cite{scikit} a Python ML library;
and \textsc{SVMMulticlass}~\cite{svmmulticlass}, a competitor to
\textsc{liblinear}. Statistics for datasets used in both parts can be found in
Fig~\ref{fig:ir}.

For the machine learning evaluation, we focus on linear classification across
toolkits. Many applications in \meta/ are in a textual domain, and linear
classification lends itself to the high-dimensional space that comes with text
documents. All experiments are performed on a system with a dual core Intel Core
i5 CPU (M460) clocked at 2.53GHz and eight gigabytes of RAM.

Four datasets (20news, Blog, Newegg, and the Yelp Academic
Dataset\footnotemark[10]) are textual datasets. The Newegg (crawled ourselves)
and Yelp datasets are review datasets, and we consider only a partial list of
the whole dataset where we have removed all three-star reviews. We used MeTA for
tokenization and feature generation, applying the same constraints as we did for
the indexing tests. Randomized training (two thirds) and test splits (one third)
were generated for each of these datasets. For rcv1, we used the existing
tokenization and training/test splits available on the LIBSVM data
website~\footnotemark[11].

\footnotetext[10]{\url{https://www.yelp.com/academic_dataset}}
\footnotetext[11]{\url{http://www.csie.ntu.edu.tw/~cjlin/libsvmtools/datasets/}}

In Fig~\ref{fig:classification}, we can see that \meta/ performs well both in
terms of speed and accuracy, and presents itself as a viable option in the
machine learning domain.

\begin{figure}
\centering
{\small
\begin{tabular}{|l|r|r|r|r|}
    \hline & \textbf{Docs} & \textbf{Size} & \textbf{$|D|_{avg}$} & \textbf{$|V|$} \\
    \hline
    \textbf{20news}  & 18,828     & 39 MB  & 223.4      & 97,667     \\
    \textbf{Blog}    & 18,713     & 654 MB & 3432.1     & 462,425    \\
    \textbf{Homes}   & 324,553    & 5.0 GB & 1742.3     & 3,609,811  \\
    \textbf{Newegg}  & 106,998    & 57 MB  & 55.6       & 38,533     \\
    \textbf{rcv1}    & 697,641    & 1.2 GB & \emph{N/A} & 47,236 \\
    \textbf{Reddit}  & 15,497,530 & 2.6 GB & 18.6       & 1,942,065  \\
    \textbf{Yelp}    & 194,543    & 130 MB & 74.9       & 75,703 \\
    \textbf{Wiki}    & 8,284,146  & 25 GB  & 272.3      & 15,248,289 \\
    \hline
\end{tabular}
}
\caption{Datasets for IR and ML experiments.}
\label{fig:datasets}
\end{figure}

\begin{figure}[t]
\centering
{\small
\begin{tabular}{|l|r|r|r|r|}
  \hline
  & \textbf{liblinear} & \textbf{scikit} &
  \textbf{SVM\textsuperscript{mult}} & \textbf{MeTA}\\\hline
  \textbf{20news}                    & 88.18\%          & 78.03\%  & 70.27\%          & \textbf{89.10\%}\\
  $(k=20)$                           & 15.64s           & 5.28s    & 35.36s           & 7.55s\\\hline
  \textbf{Blog}                      & 45.53\%          & 35.91\%  & \textbf{46.22\%} & 41.14\%\\
  $(k=7)$                            & 310.6s           & 46.94s   & 164.6s           & 28.34s\\\hline
  \textbf{Newegg}                    & 90.39\%          & 90.71\%  & 78.71\%          & \textbf{91.96\%}\\
  $(k=2)$                            & 23.90s           & 10.93s   & 3.806s           & 4.182s\\\hline
  \textbf{Yelp}                      & 91.65\%          & 92.02\%  & 80.22\%          & \textbf{92.98\%}\\
  $(k=2)$                            & 51.89s           & 27.48s   & 9.700s           & 8.854s\\\hline
  \textbf{rcv1}                      & \textbf{88.43\%} & 87.68\%  & 55.21\%          & 79.06\%\\
  $(k=2)$                            & 21.04s           & 99.07s   & 3.37s            & 37.42s\\\hline
\end{tabular}
}
\caption{Accuracy and speed classification results.}
\label{fig:classification}
\end{figure}

