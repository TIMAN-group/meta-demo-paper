\section{Usability}

Consistency across components is a key feature that allows \meta/ to work
well with large datasets. This is accomplished via a three-layer architecture.
On the first layer, we have tokenizers, analyzers, and all the text processing
that accompanies them. Once a document representation is determined, this tool
chain is run on a corpus. The indexes are the second layer; they provide an
efficient format for storing processed data. The third layer---the application
layer---interfaces solely with indexes. This means that we may use the same
index for running an SVM as we do to evaluate a ranking function, \emph{without
processing the data again}.

Since all applications use these indexes, \meta/ is able to support
out-of-core classification with some classifiers. We ran our large
classification dataset, rcv1~\cite{rcv1}, with limited memory (only 95MB) using
the \texttt{sgd} classifier.
\todo{Update info about online classification.}
Where \textsc{liblinear} failed to run,
\meta/ was able to finish the classification in just under two minutes.

Besides using \meta/'s rich built-in feature generation, it is possible to
directly use \textsc{libsvm}-formatted data. This allows preprocessed datasets
to be run under \meta/'s algorithms. Additionally, \meta/'s
\texttt{forward\_index} (used for classification), is easily convertible to
\textsc{libsvm} format. Thus the reverse is also true: you may do feature
generation with \meta/, and use it to generate input for any other program that
supports \textsc{libsvm} format.

\todo{Mention forum.}
\meta/ is hosted publicly on GitHub\footnotemark[1], which provides
the project with community involvement through its bug/issue tracker and
fork/pull request model. Its API is heavily documented\footnotemark[2] and the
project website contains several tutorials that cover the major aspects of
the toolkit\footnotemark[3] to enable users to get started as fast as
possible with little friction.

\footnotetext[1]{\url{https://github.com/meta-toolkit/meta/}}
\footnotetext[2]{\url{https://meta-toolkit.org/doxygen/namespaces.html}}
\footnotetext[3]{\url{https://meta-toolkit.org/}}

A major design point in \meta/ is to allow for most of the functionality to be
configured via a configuration file. This enables exploratory data analysis
without having to write (or recompile) any code. Designing the code in this way
also encourages the components of the system to be pluggable: the entire
indexing process, for example, consists of several modular layers which can be
controlled by the configuration file.
\todo{Insert a figure of a simple TOML file.}
\todo{Mention usage in classes and MOOCs.}

A simple class hierarchy allows users to add filters, analyzers, ranking
functions, and classifiers with full integration to the toolkit
(\emph{e.g.}\ one
may specify user-defined classes in the config file). The process for adding
these is detailed in the \meta/ online tutorials.
\todo{Mention API.}

Multi-language support is hard to do correctly. Many toolkits sidestep this
issue by only supporting ASCII text or the OS language; \meta/ supports multiple
(non-romance) languages by default, using the industry standard ICU
library\footnotemark[4]. This allows \meta/ to tokenize arbitrarily-encoded text
in many languages.

\footnotetext[4]{\url{http://site.icu-project.org/}}

Unit tests ensure that contributors are confident that their modifications do
not break the toolkit. Unit tests are automatically run after each commit and
pull request, so developers immediately know if there is an issue (of course,
unit tests may be run manually before committing). The unit tests are run in a
continuous integration setup where \meta/ is compiled and run on
Linux\footnotemark[77], Mac OS X\footnotemark[77], and Windows\footnotemark[78]
under a variety of compilers and software development configurations.

\footnotetext[77]{\url{https://travis-ci.org/meta-toolkit/meta}}
\footnotetext[78]{\url{https://ci.appveyor.com/project/skystrife/meta}}
