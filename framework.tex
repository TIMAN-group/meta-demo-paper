\section{A Unified Framework}
\label{sec:framework}

The existing environment of open source software solutions for text processing
is fragmented: there is rarely a single location for a wide variety of
algorithms. Tools tend to specialize on one particular area, and as such there
is a wide variety of tools one must sample when performing different data
science tasks. For text-mining tasks, this is even more apparent; it is
extremely difficult (if not impossible) to find tools that support both
traditional information retrieval tasks (like tokenization, indexing, and
search) alongside traditional machine learning tasks (like document
classification, regression, and topic modeling).

Data science students, researchers, and practitioners must find the appropriate
software packages for their needs and compile and configure each appropriate
tool. Then, there is the problem of data formatting---it is unlikely that the
tools all have standardized upon a single input format, so a certain amount of
``data munging'' is required. All of this detracts from the actual task at hand,
which has a marked impact on productivity.

The goal of the \meta/ project is to address these issues. In particular, we
provide a unifying framework for existing machine learning and natural language
processing algorithms, allowing researchers to quickly run controlled
experiments. We have modularized the feature generation, instance
representation, data storage formats, and algorithm implementations; this allows
for researchers, practitioners, and students to make seamless transitions along
any of these dimensions with minimal effort.

\meta/ is dual-licensed under the University of Illinois/NCSA Open Source
Licence and the MIT License in order to reach the broadest audience possible.

\begin{table*}[t]
    \begin{center}
    {\small
    \begin{tabular}{|l|c|c|c|c|c|c|c|c|}
        \hline
        & \textbf{Indri} & \textbf{Lucene} & \textbf{MALLET} &
        \textbf{LIBLINEAR} & \textbf{SVM$^{MULT}$} & \textbf{scikit} &
        \textbf{CoreNLP} & \textbf{\meta/} \\
        & \emph{IR} & \emph{IR} & \emph{ML/NLP} & \emph{ML} & \emph{ML} &
        \emph{ML/NLP} & \emph{ML/NLP} & \emph{all} \\
        \hline
        Feature generation & \checkmark & \checkmark & \checkmark & & &
        \checkmark & \checkmark & \checkmark \\
        Search & \checkmark & \checkmark & & & & & & \checkmark \\
        Classification & & & \checkmark & \checkmark & \checkmark & \checkmark &
        \checkmark & \checkmark \\
        Regression & & & \checkmark & \checkmark & \checkmark & \checkmark &
        \checkmark & \checkmark \\
        POS tagging & & & \checkmark & & & & \checkmark & \checkmark \\
        Parsing & & & & & & & \checkmark & \checkmark \\
        Topic models & & & \checkmark & & & \checkmark & & \checkmark \\
        $n$-gram LM & & & & & & & & \checkmark \\
        Word embeddings & & & \checkmark & & & & \checkmark & \checkmark \\
        Graph algorithms & & & & & & & & \checkmark \\
        Multithreading & & \checkmark & \checkmark & & & \checkmark & \checkmark
        & \checkmark \\
        \hline
    \end{tabular}
    \caption{Feature comparison of NLP, IR, and ML toolkits. Citations for all
        toolkits may be found in their respective comparison sections.}
    \label{tab:feature-comp}
    }
    \end{center}
\end{table*}

Table~\ref{tab:feature-comp} compares \meta/'s many features across various
toolkits. Due to space constraints, we only delve into the natural language
processing (NLP), information retrieval (IR), and machine learning (ML)
components in section~\ref{sec:experiments}. We briefly outline all components
here:

\textbf{Feature generation}.
\meta/ has a collection of tokenizers, filters, and analyzers that convert raw
text into a feature representation. Basic features are $n$-gram words, but other
analyzers make use of different parts of the toolkit, such as POS tag $n$-grams
and parse tree features. An arbitrary number of feature representations may be
combined; for example, a document could be represented as unigram words, bigram
POS tags, and parse tree rewrite rules. Users can easily add their own feature
types as well, such as sentence length distribution in a document.

\textbf{Search}.
The \meta/ search engine can store document feature vectors in an inverted index
and score them with respect to a query. Rankers include vector space models such
as Okapi BM25~\cite{bm25} and probabilistic models like Dirichlet prior
smoothing~\cite{zhai-lm}. A search demo is
online\footnote{\url{https://meta-toolkit.org/search-demo.html}}.

\textbf{Classification}.
\meta/ includes a stochastic gradient descent (SGD) implementation with
pluggable loss functions, allowing the creation of an SVM classifier (among
others). All-\emph{vs}-all and one-\emph{vs}-all ensemble methods for binary
classifiers allow multiclass classification. Other classifiers such as na{\"i}ve
Bayes and $k$-nearest neighbors also exist. A confusion matrix class and
significance testing framework allow evaluation and comparison of different
methods and feature representations.
% cite SVM..?
% feature scaling? too technical?

\textbf{Regression}.
Regression is implemented via SGD to predict real-valued responses from
featurized documents. Evaluation metrics such as mean squared error and $R^2$
score allow model comparison.

\textbf{POS tagging}.
\meta/ contains a linear-chain conditional random field for POS tagging and
chunking applications, learned using $\ell_2$ regularized SGD~\cite{crf}. It
also contains an efficient greedy averaged perceptron tagger~\cite{greedy}.

\textbf{Parsing}.
A fast shift-reduce constituency parser using generalized averaged perceptron
~\cite{const-parsing} is \meta/'s grammatical parser. Parse tree featurizers
implement different types of structural tree representations~\cite{structural}.
An NLP demo online presents tokenization, POS-tagging, and
parsing\footnote{\url{https://meta-toolkit.org/nlp-demo.html}}.

\textbf{Topic models}.
\meta/ can learn topic models over any feature representation using collapsed
variational Bayes~\cite{cvb}, collapsed Gibbs sampling~\cite{gibbs}, stochastic
collapsed variational Bayes~\cite{scvb}, or approximate distributed
LDA~\cite{pargibbs}.

\textbf{$n$-gram language models} (LMs).
\meta/ takes an ARPA-formatted
input\footnote{\url{http://www.speech.sri.com/projects/srilm/manpages/ngram-format.5.html}}
and creates a language model that can be queried for token sequence
probabilities or used in downstream applications like
SyntacticDiff~\cite{syndiff}.
% MPH not merged yet, mention?

\textbf{Word embeddings}.
The GloVe algorithm~\cite{glove} is implemented in a streaming framework and
also features an interactive semantic relationship demo. Word vectors can be
used in other applications as part of the \meta/ API\@.

\textbf{Graph algorithms}.
Directed and undirected graph implementations exist and various algorithms such
as betweenness centrality, PageRank, and myopic search are available. Random
graph generation models like Watts-Strogatz and preferential attachment exist.
For these algorithms see \newcite{networks}.

\textbf{Multithreading}.
When possible, all \meta/ algorithms and applications are parallelized using C++
threads to make full use of available resources.
