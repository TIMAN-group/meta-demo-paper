\section{A Unified Framework}

The existing environment of open source software solutions for machine learning
is fragmented: there is rarely a single location for a wide variety of
algorithms. Tools tend to specialize on one particular area, and as such there
is a wide variety of tools one must sample when performing different machine
learning tasks. For text-mining tasks, this is even more apparent; it is
extremely difficult (if not impossible) to find tools that support both
traditional information retrieval tasks (like tokenization, indexing, and
search) alongside traditional machine learning tasks (like document
classification, regression, and topic modeling).

This places an undue burden on researchers---not only are they required to have
a detailed understanding of the research problem at hand, but they are now
forced to understand this fragmented nature of the open source software
community, find the appropriate software packages for their needs, and compile
and configure each appropriate tool. Even when this is all done, there is the
problem of data formatting---it is unlikely that the tools all have standardized
upon a single input format, so a certain amount of ``data munging'' is now
required. All of this detracts from the actual research task at hand, which has
a marked impact on the speed with which new ideas are discovered.

The goal of the \meta/ project is to address these issues. In particular,
we provide a unifying framework for existing machine learning algorithms,
allowing researchers to quickly run controlled experiments. We have modularized
the feature generation, instance representation, data storage formats, and
algorithm implementations; this allows for researchers to make seamless
transitions along any of these dimensions with minimal effort.

We have a liberal licensing scheme for \meta/. It is dual-licensed under
the University of Illinois/NCSA Open Source Licence and the MIT License. We use
these permissive licenses deliberately to allow for \meta/'s
incorporation into commercial products: we feel that the use of a copyleft
license creates unnecessary barriers between the code we develop as researchers
and the applications of our techniques to areas outside of the research
community.

\begin{table*}[t]
    \begin{center}
    {\small
    \begin{tabular}{|l|c|c|c|c|c|c|c|c|c|}
        \hline
        & Indri & Lucene & LIBSVM & MALLET & scikit & Weka & Lingpipe &
        CoreNLP & \meta/ \\
        \hline
        Feature generation & & & & & & & & & \checkmark \\
        Search & & & & & & & & & \checkmark \\
        Classification & & & & & & & & & \checkmark \\
        Regression & & & & & & & & & \checkmark \\
        POS tagging & & & & & & & & & \checkmark \\
        Parsing & & & & & & & & & \checkmark \\
        Topic models & & & & & & & & & \checkmark \\
        Language models & & & & & & & & & \checkmark \\
        Word embeddings & & & & & & & & & \checkmark \\
        Graph algorithms & & & & & & & & & \checkmark \\
        \hline
    \end{tabular}
    \caption{Feature comparison of NLP, IR, and ML toolkits.}
    }
    \end{center}
\end{table*}
